% LaTeX file for resume 
% This file uses the resume document class (res.cls)

\documentclass{res} 
\usepackage{CJK}
%\usepackage{helvetica} % uses helvetica postscript font (download helvetica.sty)
%\usepackage{newcent}   % uses new century schoolbook postscript font 
\setlength{\textheight}{9.5in} % increase text height to fit on 1-page 

\begin{document} 
\begin{CJK}{UTF8}{gbsn}
\name{于鹏飞\\[5pt]}     % the \\[12pt] adds a blank
				        % line after name      

%\address{yupengfei@mail.nankai.edu.cn\\+86 189 2001 3625}

                                  
\begin{resume}

\section{个人信息} 
出生日期:1991/01/27\\
电子邮箱:yupengfei@mail.nankai.edu.cn\\
移动电话:+86 189 2086 6903
\section{求职意向}          
    软件研发         
 
\section{相关技能}          
    熟悉linux下开发环境,熟悉Geany、Qt Creator等IDE,能使用Vim文本编辑器\\
    熟悉python,能编写简单的脚本程序\\
    精通C、C++,精通QT,了解版本控制工具git,编写过简单的音乐播放器等小型软件   \\  
    掌握算法、网络、数据库、操作系统的相关知识       \\
    掌握计算软件maple、maxima,掌握排版语言\LaTeX{}\\    
    熟悉物理学大部分领域,掌握高等数学、数学物理方法、概率与统计理论、复变函数、实变函数、群论、李群等数学理论\\
    英语水平良好,一次性通过四六级,并能阅读计算机、物理领域的相关文献
%    大学英语四级 537 大学英语六级 517\\
             

 
\section{教育科研经历}
   \vspace{-0.1in}	
   \begin{tabbing}
   \hspace{2.3in}\= \hspace{2.6in}\= \kill % set up two tab positions
    {\bf 南开大学} \>物理科学学院     \>2010.9--今\\
                             \>理论物理(场论)
   \end{tabbing}\vspace{-20pt}      % suppress blank line after tabbing
    2010年南开大学张舜尧助新金\\
    参与国家自然科学基金委员会项目,编号10625521,10721063,10705001,科技部项目,
编号2009\\CB825200,教育部项目(ANEDD编号200924,DPFIHE编号20090211120029,
NCET编号NCET-10-0442),论文有\\
    The molecular systems composed of the charmed mesons in the $H\bar{S}+h.c.$ doublet\\ 		    
    Eur.Phys.J.C70:183-217,2010 \\
    Possible heavy molecular states composed of a pair of excited charm-strange mesons\\
    Chin.Phys.C35:113-125,2011\\
%    现在从事力学体系分析的计算机程序编写\\
    参与国家自然科学基金委员会项目,项目编号10875059。论文有\\
    Hamiltonian description of singular Lagrangian systems with spontaneously broken time translation symmetry\\
    http://arxiv.org/abs/1206.2983(被诺贝尔奖得主Frank Wilczek引用)\\
    Landau meets Newton: time translation symmetry breaking in classical mechanics\\
    http://arxiv.org/abs/1208.5974
    \vspace{-0.1in}
   \begin{tabbing}
   \hspace{2.3in}\= \hspace{2.6in}\= \kill % set up two tab positions
    {\bf 兰州大学} \>物理科学与技术学院 \> 2006.9--2010.6\\
                          \>隆基班(物理)
   \end{tabbing}\vspace{-20pt}
    2006-2007学年兰州大学一等奖学金\\
    2006-2007学年兰州大学三好学生\\
    2006-2007学年兰州大学基地二等奖学金\\
    物理系 86 级校友励志奖学金 三等奖\\
    2007-2008学年兰州大学三等奖学金\\
    兰州大学2008-2009学年优秀学生二等奖学金\\
    2007年兰州大学暑假社会实践先进个人\\
    2008高教社杯 全国大学生数学建模竞赛 甘肃赛区甲组二等奖\\
    获得优秀毕业论文
      
\section{兴趣爱好}          
    跆拳道 WTF二级\\
   长跑\quad\,  2007年化学化工学院 定向越野比赛优秀奖          
 
\end{resume}
\end{CJK}
\end{document}
